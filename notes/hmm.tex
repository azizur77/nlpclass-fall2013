\documentclass[11pt,letterpaper]{article}
\topmargin -.5truein
\textheight 9.0truein
\oddsidemargin 0truein
\evensidemargin 0truein
\textwidth 6.5truein
\setlength{\parskip}{5pt}
\setlength\parindent{0pt}
\usepackage[table]{xcolor}% http://ctan.org/pkg/xcolor
\usepackage{caption}
\usepackage{subcaption}
\usepackage[round]{natbib}
\usepackage{graphicx}
\usepackage{listings}
\usepackage{amsmath}
\usepackage{amssymb}
\usepackage{qtree}
\usepackage{algorithmic}
\usepackage{wrapfig}
\usepackage{tikz}
\usepackage{dsfont}
\usepackage{multirow}
\usepackage[all]{xy}
\usetikzlibrary{arrows,snakes,backgrounds,patterns,matrix,shapes,fit,calc,shadows,plotmarks}

\newcommand{\bs}{\textbackslash}
\renewcommand{\vec}[1]{\mathbf{#1}}

\newcommand{\ngramstart}{\ensuremath{\langle S \rangle}}
\newcommand{\ngramend}{\ensuremath{\langle E \rangle}}
\newcommand{\ngramunk}{\ensuremath{\langle unk \rangle}}

\usepackage{hyperref}
\hypersetup{
    colorlinks,
    citecolor=black,
    filecolor=black,
    linkcolor=black,
    urlcolor=black
}

\title{NLP: Hidden Markov Models}
\author{Dan Garrette\\\small{dhg@cs.utexas.edu}}

\begin{document}
\maketitle


Example dataset:
\vspace{-2mm}
\begin{verbatim}
    <S>|<S> the|D dog|N runs|V   .|.   <E>|<E> 
    <S>|<S> the|D dog|N walks|V  .|.   <E>|<E> 
    <S>|<S> the|D man|N walks|V  .|.   <E>|<E> 
    <S>|<S> a|D   man|N walks|V  the|D dog|N   .|. <E>|<E> 
    <S>|<S> the|D cat|N walks|V  .|.   <E>|<E> 
    <S>|<S> the|D dog|N chases|V the|D cat|N   .|. <E>|<E> 
\end{verbatim}



\newsavebox{\starttable}
\sbox{\starttable}{
    \begin{tabular}{l l}
      \hline
      \multicolumn{2}{|c|}{{\cellcolor{gray!40}\ngramstart}} \tabularnewline
      \hline
      \ngramstart & 1.0
    \end{tabular}
}

\newsavebox{\Dtable}
\sbox{\Dtable}{
    \begin{tabular}{l l}
      \hline
      \multicolumn{2}{|c|}{{\cellcolor{gray!40}D}} \tabularnewline
      \hline
      the & 0.87 \tabularnewline
      a   & 0.13
    \end{tabular}
}

\newsavebox{\Ntable}
\sbox{\Ntable}{
    \begin{tabular}{l l}
      \hline
      \multicolumn{2}{|c|}{{\cellcolor{gray!40}N}} \tabularnewline
      \hline
      man & 0.25 \tabularnewline
      dog & 0.50 \tabularnewline
      cat & 0.25
    \end{tabular}
}

\newsavebox{\Vtable}
\sbox{\Vtable}{
    \begin{tabular}{l l}
      \hline
      \multicolumn{2}{|c|}{{\cellcolor{gray!40}V}} \tabularnewline
      \hline
      walks  & 0.66 \tabularnewline
      runs   & 0.17 \tabularnewline
      chases & 0.17
    \end{tabular}
}

\newsavebox{\stoptable}
\sbox{\stoptable}{
    \begin{tabular}{l l}
      \hline
      \multicolumn{2}{|c|}{{\cellcolor{gray!40}.}} \tabularnewline
      \hline
      . & 1.0
    \end{tabular}
}

\newsavebox{\ngramendtable}
\sbox{\ngramendtable}{
    \begin{tabular}{l l}
      \hline
      \multicolumn{2}{|c|}{{\cellcolor{gray!40}\ngramend}} \tabularnewline
      \hline
      \ngramend & 1.0
    \end{tabular}
}


\begin{figure}
\centering
\begin{tikzpicture}
\tikzstyle{main}=[rectangle, thick, draw =black!80, node distance = 40mm]
\tikzstyle{obsv}=[main, fill = black!10]
\tikzstyle{hidden}=[node distance = 16mm]
\tikzstyle{connect}=[-latex, thick]
\tikzstyle{box}=[rectangle, draw=black!100]

  \node[main] (start) []           {\usebox{\starttable}};
  \node[main] (D)     [right of =start]          {\usebox{\Dtable}};
  \node[main] (N)     [right of =D, yshift= 70]  {\usebox{\Ntable}};
  \node[main] (V)     [right of =D, yshift=-70]  {\usebox{\Vtable}};
  \node[main] (stop)  [right of =D, xshift=40mm] {\usebox{\stoptable}};
  \node[main] (end)   [right of =stop]           {\usebox{\ngramendtable}};


  \path (start) edge [connect] node [pos=0.5, above, blue, sloped] {1.0} (D);
  \path (D) edge [connect] node [pos=0.5, above, blue, sloped] {1.0} (N);
  \path (N) edge [connect] node [pos=0.5, above, blue, sloped] {0.80} (V);
  \path (N) edge [connect] node [pos=0.5, above, blue, sloped] {0.20} (stop);
  \path (V) edge [connect, red] node [pos=0.5, above, blue, sloped] {0.33} (D);
  \path (V) edge [connect] node [pos=0.5, above, blue, sloped] {0.67} (stop);
  \path (stop) edge [connect] node [pos=0.5, above, blue, sloped] {1.0} (end);
  

\end{tikzpicture}
\caption{Finite state machine.  Missing arrows are assumed to be zero probabilities.  With smoothing, there would be an arrow in \textit{both directions} between \textit{every} pair of words.}
\end{figure} 






\section{Citations}

Some content adapted from:

\begin{itemize}
  \item \url{http://courses.washington.edu/ling570/gina_fall11/slides/ling570_class8_smoothing.pdf}
\end{itemize}


\end{document}

